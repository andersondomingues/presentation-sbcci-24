\section{Introduction}

\begin{frame}{Introduction}
    \begin{itemize}
        \item Manycore systems are intrinsic to modern interconnected environments, particularly in IoT networks. 
        
        \item However, this interconnectivity renders them \textbf{susceptible to various forms of cyberattacks}.
        
        \item The use of third-party intellectual property cores as a solution to cope with design complexity and time-to-market pressures further worsens these vulnerabilities.
        
        \item Security is a critical concern in developing and deploying manycore systems.

        \item The existing literature offers various strategies for threat \ul{\textit{detection}} \cite{Charles:2021:Survey}, \ul{\textit{localization}} of these threats \cite{Subodha:2020:Localization}, and the corresponding \ul{\textit{countermeasures}} \cite{Faccenda:2021:Countermeasures} in NoC-based manycores. This work focuses on \textbf{threat detection}.  
    \end{itemize}
\end{frame}

\begin{frame}{Related Work}
    
    \begin{itemize}
    
    \item Early research: ad-hoc algorithms based on exclusive latency monitoring, such as RLAN \cite{Rajesh2015}, that cannot account for variations such as in application mapping.
    
    \item Kulkarni et al. \cite{Kulkarni:2016:SVM} presents one of the first ML proposals for anomaly detection in manycore systems. The Authors assess diverse ML algorithms and choose SVM as their implementation in FPGA. This implementation employs synthetic traffic generated by an external data injector, which does not accurately represent real-world workloads. 
                    
    \end{itemize}
\end{frame}

\begin{frame}{Related Work}
    
    \begin{itemize}
                    
    \item Sudusinghe et al. propose methods for detecting flooding DoS \cite{Sudusinghe:2021:DoS} and eavesdropping attacks \cite{Sudusinghe:2022:Eevesdropping} using ensemble learners. Routers in their systems have probes that send relevant features to centralized detectors. The Authors evaluate their work with Gem5.

    \item TSA-NoC \cite{Wang:2020:TSA-NoC} and AGAPE \cite{Wang:2022:AGAPE} detect HTs using Neural Networks (NN). These approaches have as a drawback the NN area overhead, which every NoC router incorporates.
    
    \end{itemize}
\end{frame}


\begin{frame}{Problem}
    \begin{itemize}

    \item Limitations found in early work on threat localization and in works using ML.

    \begin{itemize}
        \item \ul{\textit{applicability}}, is due to costly techniques, usually based on NN, replicated at every router in the NoC \cite{Rajesh2015, Wang:2020:TSA-NoC, Wang:2022:AGAPE, Sinha:2021:Sniffer}. 

        \ul{\textit{confidence}}, is due to the adoption of synthetic NoC traffic that does not represent real application scenarios \cite{Kulkarni:2016:SVM, Madden:2018:NN, Vashist:2019:WiNoC, Yao:2020:Localization, Hu:2023:Cascaded} and the use of high-level simulators \cite{Sudusinghe:2021:DoS, Sudusinghe:2022:Eevesdropping, Wang:2020:TSA-NoC, Wang:2022:AGAPE, Sinha:2021:Sniffer, Madden:2018:NN, Yao:2020:Localization, Hu:2023:Cascaded}, such as Gem5, which do not reflect the actual accuracy of the systems.
    \end{itemize}
        
    \end{itemize}
\end{frame}

\begin{frame}{Proposed solution}
    \begin{itemize}    
    
    \item Goals: \begin{itemize}
        \item detect anomalies in the NoC traffic caused by malicious sources, characterizing a security threat
        \item using an ML flow that profiles real applications in an actual manycore, modeled at the RTL level with clock cycle accuracy.
    \end{itemize}

    \item Original contributions \begin{itemize}
        \item create and use application profiles instead of synthetic NoC traffic

        \item detect subtle, uncorrelated traffic collisions that occur due to disturbing malicious traffic in applications.
    \end{itemize}
    
    \end{itemize}
\end{frame}
