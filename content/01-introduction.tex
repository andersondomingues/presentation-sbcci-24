\section{Introduction}

\begin{frame}{Introduction}
    \begin{itemize}
        \item NoCs are attractive options for system interconnection in
        manycores due to their potential for massively parallel communication and scalability while inserting low overhead for area and energy consumption into the design. 
        
        \item Routers provide the communication infrastructure in a typical NoC, allowing multiple data streams to simultaneously traverse the system  \cite{kumar:2021, benini:2002, jantsch:2005}. 
        
        \item Some NoC designs acknowledge the implementation of NFR, e.g., security, safety~\cite{Charles:2020, Guo:2020}, and real-time \textcolor{red}{(*)}. 
        
        \item The main concern of RT-NoCs is predictability --- a particularly important property in control-theoretic, cyber-physical, and robotics domains due to the need for compliance with stringent safety constraints~\cite{autosar:2021, iso:asil:2020, spice:2021}.%GRAMMARLY-OK
        
    \end{itemize}
\end{frame}

\begin{frame}{Introduction (contd.)}
	\begin{itemize}
		\item Designing hardware that accounts for multiple NFRs is challenging due to the implementation of \textcolor{red}{conflicting NFRs}, e.g., implementing a security feature may modify the timing* of the system. Similarly, predictable hardware may add energy and area overheads, due to additional circuitry.
		
		\item The lack of specialized literature aggravates the problem, as the  techniques often manage NFRs as isolated. 
		
		\item Also, studies on multiple NFRs, e.g., power and real-time~\cite{li:2023} propose techniques on non-publicly available platforms, making it unfeasible to replicate the techniques on industrial-grade projects.
		
	\end{itemize}
\end{frame}

\begin{frame}{Concerns of RT Analysis on Manycores}
	\begin{itemize}
		\item The analysis of applications on NoC-based, private memory, and multitasking manycores must consider both the individual processing elements (PEs) and the underlying NoC. The analysis of PEs regards the CPU and I/O peripherals, while RT-NoCs partake in the RT guarantees for communication. 
		
		\item Recent studies on RT-NoCs~\cite{Deniziak:2018, chen:2020, Picornell:2019} do not approach computation, missing mechanisms for synchronizing processing and communication operations. 
		
		\begin{itemize}
		\item Considering commercial off-the-shelf (COTS) platforms, whose hardware cannot be changed, is it possible to run a given application and guarantee RT for tasks and traffic? 
		
		\item Assuming platforms where we can change the nominal frequency, what is the minimum frequency to address the RT requirements of an application? How can we address RT without tampering with other NFRs?
		\end{itemize}
	\end{itemize}
\end{frame}


\begin{frame}{This paper...}
	\begin{itemize}
		\item Addresses RT analysis on NoC-based, multitasking, private memory manycores whose hardware cannot be changed, e.g., COTS platforms, targeting manycores equipped with conventional, non-RT NoCs. 
		
		\item As these platforms cannot take advantage of specific design-time optimizations~\cite{Esperanto,Bohnenstiehl}, we propose an off-line, \textit{pre-runtime} framework for computation and communication RT analysis that can:%GRAMMARLY-OK
		
		\begin{enumerate}
			%, not interfering with the hardware;
			\item determine if a given application meet its RT requirements, considering the architectural features and nominal operating frequency of the platform;
			\item find the minimum frequency to address an RT application in platforms whose frequency can be changed.   
		\end{enumerate}
		
	\end{itemize}
\end{frame}

\begin{frame}{This paper... (contd.)}
		
	\begin{itemize}
	
		\item At the computation scope, we combine a discrete-event simulation (DES) strategy with the critical-path analysis (CPA). 
		
		\item The goal is to validate the CPU requirements for tasks at the PE level and the end-to-end application execution time at a system-wide level. 
		
		\item We adopt a contention-free model for communication, assuming a pre-runtime optimization method. 
		
		\item We assume the zero-load latency model of the NoC as a performance boundary. 
		
		\item This assumption enables us to perform our analysis on \emph{virtually any NoC}, adding RT support to systems without RT-NoCs. 
		
		\item Our approach is \emph{novel} because it supports COTS, is fully automated, and uses only open-source tools. We validate our approach in an open-hardware platform, favoring the reproducibility of our study.%GRAMMARLY-OK

	\end{itemize}	
\end{frame}